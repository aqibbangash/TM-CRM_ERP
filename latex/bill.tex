\documentclass[a4paper, oneside, 10pt, french]{article}

\usepackage[a4paper]{geometry}
\geometry{top=1.5cm, bottom=1cm, left=1.5cm, right=1.5cm,foot=.2cm,head=2cm,headsep=.5cm,includefoot}

\usepackage{tm_article}
\usepackage{tabularx, longtable, multirow, rotating}

\definecolor{violet}{rgb}{0.0,0.0,0.0}
\definecolor{bleu}{rgb}{0.01,0.28,0.58}
\newcommand{\vhline}{\arrayrulecolor{violet}\hline\arrayrulecolor{black}}
\newcommand{\ghline}{\arrayrulecolor{gray}\hline\arrayrulecolor{black}}
\newcommand{\bhline}{\arrayrulecolor{bleu}\hline\arrayrulecolor{black}}
\makeatletter
\renewcommand{\section}{\@startsection {section}{1}{\z@}%
             {-0ex \@plus -1ex \@minus -.2ex}%
             {0.01ex \@plus.1ex}%
             {\footnotesize\sffamily}}
\makeatother

\newcommand{\FOOT}{--FOOT--}
\renewcommand{\HEAD}{\directlua{tex.print(json.title.value)} \directlua{tex.print(json.ref.value)}}

\newif\ifdiscount

\discountfalse

\directlua{
 if json.isDiscount.value then
	 tex.sprint([[\noexpand\discounttrue]])
 end
}

\begin{document}

\begin{tabular}{p{9cm} p{8cm}}
    \vspace{0pt} 
    \includegraphics[width=7.5cm]{--LOGO--}
    & 
    \vspace{0pt}
   \raggedleft
	%\begin{flushright}
	\textcolor{violet}{\textsc{\Large \directlua{tex.print(json.title.value)} \directlua{tex.print(json.ref.value)}}}\\
	Date de facture :  \directlua{tex.print(json.datec.value)}\\ \directlua{tex.print(json.period.value)}
	Ref. commande client :  \directlua{tex.print(json.ref_client.value)}\\
	TVA Intra. du  client : \directlua{tex.print(json.to.value.tva or " ")}~\\
\end{tabular}

%\vspace{-1em}

\begin{minipage}[t]{0.40\textwidth}
{\small Emetteur :}\\
\begin{fminipage}
--MYSOC--
\end{fminipage}
\end{minipage}
\hspace{1cm}
\begin{minipage}[t]{0.52\textwidth}
{\small Destinataire :}

\begin{fminipage}
\textbf{\large \directlua{tex.print(json.to.value.name)}}\\
\directlua{tex.print(json.to.value.address.street)}\\
\textsc{\directlua{tex.print(json.to.value.address.zip)} \directlua{tex.print(json.to.value.address.city)}}\\
\begin{minipage}{\textwidth}
\flushright
{\tiny \directlua{tex.print(json.to.value.code_client)}}
\end{minipage}
\end{fminipage}
\end{minipage}

\begin{minipage}[t]{0.60\textwidth}
{\small \it Informations :}\\
\directlua{tex.print(json.notes.value)} \\
\end{minipage}
\hspace{1cm}
\begin{minipage}[t]{0.32\textwidth}
\begin{flushright}
\textit{Date d'échéance : \textbf{\directlua{tex.print(json.dateech.value)}}}\\
\vspace{1em}
{\footnotesize \textit{Montants exprimés en \euro}}
\end{flushright}
\end{minipage}

\input{lines}

\begin{minipage}[t]{0.49\textwidth}
\begin{fminipage}
{\footnotesize \textbf{Conditions de règlement :} \directlua{tex.print(json.reglement.value)}}\\
{\footnotesize Règlement TTC par \textbf{\directlua{tex.print(json.paid.value)}} \directlua{tex.print(json.bank.value)}}\\
{\it \footnotesize --VATMODE--}
\end{fminipage}
\end{minipage}
\hspace{.1cm}
\begin{minipage}[t]{0.50\textwidth}
\begin{flushright}
\begin{tabular}{|p{4.5cm} r|}

\hline
\directlua{printTotal()}
\hline
\rowcolor{violet}\textcolor{white}{NET à payer} &\textcolor{white}{\directlua{tex.print(json.APAYER.value)}} \\ 
\hline
\end{tabular} 
\end{flushright}
%\begin{flushright}
%\begin{footnotesize}
%Versements déjà effectués :
% \begin{tabular}{p{1.8cm} c p{1.8cm} c}
%Règlement & Montant & Type & Numéro \\ 
%\ghline 
% --DATEACPTE--& --ACPTE-- & --MODEREG-- & --NUMP-- \\ 
%\ghline 
%\end{tabular}
% \end{footnotesize} 
%\end{flushright}
\end{minipage}


\end{document}

